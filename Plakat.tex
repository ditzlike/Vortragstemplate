\documentclass[department=FakIM,parskip=half]{OTHR_Placard}

\usepackage[utf8]{inputenc}
\usepackage[ngerman]{babel}

\setbackground{FSIMHintergrund}
\begin{document}

\begin{placard}

\begin{psection}{Vortrag}
\huge C++ ist blöd.\\[.5ex] \scriptsize Algorithmen-Varianten in High-Performance Anwendungen,  Zero-Cost Abstractions und, wenn die Zeit reicht, ein unpopulärer  Lösungsansatz :)
\end{psection}

\begin{psection}{Vortragender}
\vspace*{1ex}
{\huge Kai Selgrad}
\end{psection}

\begin{psection}{Termin}
\huge Dienstag, 28.\,Juni\,2022
\end{psection}

\begin{psection}{Zeit}
\LARGE 19:00 Uhr
\end{psection}

\begin{psection}{Ort}
\LARGE Galgenbergstraße 32, K\,014
\end{psection}


%A\,N\,M\,E\,L\,D\,U\,N\,G
%
%{\vspace{2ex}
%\raggedright \scriptsize  Bundesnetzagentur, Außenstelle Dortmund \\
%  Dienstleistungszentrum 10 \\
%  Alter Hellweg 56 \\
%  44379 Dortmund \\[1ex]
%  Formulare: http://www.bundesnetzagentur.de/amateurfunk
%  
%
%{}\hfill\begin{picture}(0,0)\put(-25,+15){\setlength{\fboxsep}{2mm}\fcolorbox{white}{white}{\color{black}{\qrcode{http://www.bundesnetzagentur.de/amateurfunk}}}}
%\end{picture}

%}
\end{placard}



\end{document}
